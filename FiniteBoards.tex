\documentclass[11pt]{article}
\usepackage{graphicx}    % needed for including graphics e.g. EPS, PS
\topmargin -1.5cm        % read Lamport p.163
\oddsidemargin -0.04cm   % read Lamport p.163
\evensidemargin -0.04cm  % same as oddsidemargin but for left-hand pages
\textwidth 16.59cm
\textheight 21.94cm 
%\pagestyle{empty}       % Uncomment if don't want page numbers
\parskip 7.2pt           % sets spacing between paragraphs
%\renewcommand{\baselinestretch}{1.5} % Uncomment for 1.5 spacing between lines
\usepackage{amsmath}
\usepackage{amsthm}
\usepackage{amsfonts}
\usepackage{verbatim}
\newtheorem{theorem}{Theorem}
\newtheorem{lemma}{Lemma}
\newtheorem{definition}{Definition}
\parindent 0pt		 % sets leading space for paragraphs
\author{Perry Kleinhenz, Fermi Ma, and Erik Waingarten}
\title{18.821 - Poker Chips Finite board configurations}


\begin{document}         
\maketitle

In this file, we analyze the case where the initial number of chips is finite.
That is, the sum of all the chips on the board is finite. 

\section{Preliminaries}
\begin{definition} A board is a map from the integer lattice to the non-negative integers.
That is a Board $f$ is an element of $\mathcal{B} := \text{Hom}(\mathbb{Z} \times \mathbb{Z}, \mathbb{Z}_{\geq 0})$.
The input values of a board represent the coordinates on the integer lattice that the lower left hand corner of square occupies and the output value refers to the number of chips on that square. 
A square is an element of $\mathbb{Z} \times \mathbb{Z}$. 
\end{definition}

\begin{definition} We define the big earthquake operator as a map from the set of all boards to the set of all boards and write $BE: \mathcal{B} \rightarrow \mathcal{B}$. 
It transforms all of the squares on the board with the following rule.
\begin{equation}
BE(f(x,y)) = f(x,y) - 4[[f(x,y) \geq 4]] + [[f(x-1,y) \geq 4]] + [[f(x+1,y) \geq 4]] + [[f(x,y-1) \geq 4]] + [[f(x,y+1) \geq 4]],
\end{equation}
where $[[A]]$ is defined as 
\begin{equation}
[[A]] = 
\begin{cases} 
1 \text{ if } A \text{ is true} \\ 
0 \text{ if } A  \text{ is false}
 \end{cases}.
\end{equation}
\end{definition}

\begin{definition}
 We define the small earthquake operator as a map from the set of all boards to the set of all boards and write $SE: \mathcal{B} \rightarrow \mathcal{B}$.
It selects one square $(x,y)$ with $f(x,y) \geq 4$ and transforms it and all of its neighbors with the following rule: 
\begin{align*}
SE( f(x,y)) = f(x,y)-4 \\
SE( f(x+1,y)) = f(x+1,y)+1 \\
SE( f(x-1,y)) = f(x-1,y)+1 \\
SE( f(x,y+1)) = f(x,y+1)+1 \\
SE( f(x,y-1)) = f(x,y-1)+1,
\end{align*}
and leaves all other squares unchanged.
\end{definition}

\begin{definition}
We say that a board is stable if its values are strictly less than 4. This is equivalent to both of the earthquake maps having no effect on the board. 
\end{definition}

\begin{definition} 
We say that a board is finite if the total number of chips on it is finite. That is a board $f$ is finite if the sum
\begin{equation}
S= \sum_{(x,y) \in \mathbb{Z} \times \mathbb{Z}} f(x,y) 
\end{equation}
is finite. 
We denote the set of finite boards as $\mathcal{B}_f$.
\end{definition}

\begin{definition}
We define the function $n: \mathbb{Z} \times \mathbb{Z} \rightarrow \mathbb{Z}^2 \times \mathbb{Z}^2 \times \mathbb{Z}^2 \times \mathbb{Z}^2 $ as 
\begin{equation}
n(x,y) = \{ (x+1, y), (x-1, y), (x, y-1), (x, y+1) \}
\end{equation}
This function takes in a square and returns its neighbors. 
\end{definition}

\begin{theorem}
\label{finitestability}
If the initial board has a finite number of chips, there exists an $N$ and $M$ such that after $N$ big earthquakes or $M$ small earthquakes the resultant board will be stable.
\end{theorem}

We will first show that finite boards cannot return to a previous state. We will then show that for a given finite board there is a maximum nunmber of states that a board can occupy. This shows that all finite boards 

So a restatement of the theorem is to say that for all $B \in \mathcal{B}_f$ there exists an $N$ and $M$ such that for all $m \in \mathbb{Z}$, $BE^N(f) = BE^{N+m}(f)$ and for all $k \in \mathbb{Z}$, $SE^{M}(f) = SE^{M+k}(f)$

\begin{lemma}
If $BE(f) \neq f$, then $BE^n(f) \neq f$ for all $n > 0$.
\end{lemma}

\begin{proof}
Let $\Phi: \mathcal{B}_f \rightarrow \mathbb{Z}_{\geq 0}$ be the following function
\[ \Phi(f) = \sum_{x,y} f(x,y)(|x|+|y|)^2 \]
Note that this is well-defined since the boards contain a finite number of chips. We will show that 
\[ BE(f) \neq f \Rightarrow \Phi(BE(f)) > \Phi(f) \]
This means that if there is an earthquake that changes things, then $\Phi(f)$ increases, which means that if $E(f) \neq f$, then for all $N > 0$, $E^N(f) \neq f$, or that you never return to your original state.

We can evaluate $\Phi(BE(f)) - \Phi(f)$.
\[ \Phi(BE(f)) = \sum_{x,y} BE(f(x,y)) (|x|+|y|)^2 \]
\[ = \sum_{x,y} \left(f(x,y) + \sum_{l \in n(x,y)} [[f(t) \geq 4]] - 4[[f(x,y) \geq 4]]\right) (|x|+|y|)^2 \]
Suppose location $f(x,y) \geq 4$, then the number of chips in location $(x,y)$ will have decreased after applying $E$. For each decrease in the number of chips atlocation $(x,y)$ of $4$, there is an increase in its four neighbors neighbors by $1$. 

Let $A$ be the set of active locations, or locations which $f(x,y) \geq 4$, which is non-empty by assumption. 
So
\[ \Phi(BE(f)) - \Phi(f) \geq \sum_{(x,y) \in A} (-4)(|x|+|y|)^2 + 2(|x|+|y|+1)^2 + 2(|x|+|y|-1)^2 \]
\[ \geq \sum_{(x,y) \in A} \begin{array}{cccccc} -4x^2 & -4y^2 & -8xy & & &\\
									   +2x^2 & +2y^2 & +4xy & +2x& +2y &+ 2\\
									   +2x^2 & +2y^2 & +4xy & -2x& -2y  & +2 \end{array} \]
\[ \geq 4 > 0 \] 
\end{proof}

\begin{lemma}
If $SE(f) \neq f$, then $SE^n(f) \neq f$ for all $n > 0$.
\end{lemma}
Our proof of this lemma is essentiall the same as the preceeding one. We construct a potential function and show that if a small earthquake acts nontrivially on a board then the value of the potential function increases. 

\begin{proof}
We use the same definition for $\Phi: \mathcal{B}_f \rightarrow \mathbb{Z}_{\geq 0}$ 
\[ \Phi(f) = \sum_{x,y} f(x,y)(|x|+|y|)^2, \]
and evaluate $\Phi(SE(f)) - \Phi(f)$. Suppose $SE$ effected square $(x_0,y_0)$ and its neighbors, then
\begin{align}
\Phi(SE(f)) - \Phi(f) =\sum_{x,y} SE(f(x,y))(|x|+|y|)^2 - \sum_{x,y} f(x,y)(|x|+|y|)^2 \\
= SE(f(x_0,y_0))(|x_0|+|y_0|)^2 + \sum_{(x,y) \in n(x_0,y_0)} SE(f(x,y))(|x|+|y|)^2 \\- f(x_0,y_0)(|x_0|+|y_0|)^2 + \sum_{(x,y) \in n(x_0,y_0)} f(x,y)(|x|+|y|)^2 \\
= 4(|x_0| + |y_0|)^2 +\sum_{(x,y) \in n(x_0, y_0)} (|x| + |y|)^2
\end{align}
And because at least two of the neighbors of $(x_0, y_0)$ are one unit farther from the origin than $(x_0, y_0)$ and at most two of the neighbors are one unit closer, and $(|x_0|+|y_0| +1)^2 > (|x_0|+|y_0| - 1)^2$, we can see,
\begin{align}
4(|x_0| + |y_0|)^2 +\sum_{(x,y) \in n(x_0, y_0)} (|x| + |y|)^2  \\ 
\geq 4(|x_0| + |y_0|)^2 + 2 (|x_0| + |y_0|+1)^2 + 2 (|x_0| + |y_0|-1)^2 = 4 >0
\end{align}
\end{proof}


\begin{lemma}
\label{finiteextension}
Suppose $\sum_{x,y} f(x,y) = S$, and all chips are within the range $[x_l, x_h] \times [y_l, y_h]$. Then $E^n(f)$ only contains chips within the range $[x_l - \sqrt{S}, x_h + \sqrt{S}] \times [y_l-\sqrt{S}, y_h+\sqrt{S}]$ for all $n$.
\end{lemma}

\begin{proof}
Let $A$ be an initial region of the board
\end{proof}

So now we are ready to prove the theorem that all finite boards will reach a stable state. 

\begin{proof}
(of Theorem~\ref{finitestability}) Suppose $f \in \mathcal{B}$ is a finite board, so $\sum_{x,y} f(x,y) = S$, and suppose all chips lie within the locations $[x_l, x_h] \times [y_l, y_h]$. Then by Lemma~\ref{finiteextension}, we have that all $E^n(f)$ will lie in $R = [x_l - \log S, x_h + \log S] \times [y_l - \log S, y_h + \log S]$. 

There exists less than $(x_h - x_l + 2\log S)(y_h - y_l + 2\log S)S$ possible boards $b \in \mathcal{B}$ that satisfy this. We know that there are no cycles in the states, so there must be an $0 \leq N < (x_h - x_l + 2\log S)(y_h - y_l + 2\log S)S$ such that $E^N(f) = E^{N+m}(f)$ for all $m \geq 0$.
\end{proof}

\end{document}
