\documentclass[11pt]{article}
\usepackage{graphicx}    % needed for including graphics e.g. EPS, PS
\topmargin -1.5cm        % read Lamport p.163
\oddsidemargin -0.04cm   % read Lamport p.163
\evensidemargin -0.04cm  % same as oddsidemargin but for left-hand pages
\textwidth 16.59cm
\textheight 21.94cm 
%\pagestyle{empty}       % Uncomment if don't want page numbers
\parskip 7.2pt           % sets spacing between paragraphs
%\renewcommand{\baselinestretch}{1.5} % Uncomment for 1.5 spacing between lines
\usepackage{amsmath}
\usepackage{amsthm}
\usepackage{amsfonts}
\usepackage{verbatim}
\newtheorem{theorem}{Theorem}
\newtheorem{lemma}{Lemma}
\newtheorem{definition}{Definition}
\parindent 0pt		 % sets leading space for paragraphs
\author{Perry Kleinhenz, Fermi Ma, and Erik Waingarten}
\title{18.821 - Poker Chips}


\begin{document}         
\maketitle

\begin{definition}
Let a \textbf{valid} earthquake be an earthquake that occurs on a square that, at the time of the earthquake, has at least 4 chips.
\end{definition}
\begin{theorem}
For any board with finitely many chips that stabilizes after finitely many earthquakes, the order in which the earthquakes occur does not matter.
\end{theorem}
\begin{proof}
The general idea is that we can recursively specify a sequence of earthquakes that ``must" happen, and that we can move it to the front of the sequence of earthquakes without changing the final result. This technique will then uniquely (up to a point) specify an order in which the earthquakes will occur, which completes the proof.

First, we need a lemma in order to do the rearrangements.

\begin{lemma}
Let $E_1,E_2,E_3,\dots, E_k$ be a sequence of earthquakes, where each $E_i$ specifies the location of a square hit. Then the sequence given by moving some earthquake $E_j$ forward in the sequence to a new spot where $E_j$ is still a valid earthquake gives the same final configuration.
\end{lemma}
\begin{proof}
Will need to formalize this, but this is true because addition and subtraction are commutative, and that moving $E_j$ earlier up in the sequence can't hurt another other squares' ability to have an earthquake.
\end{proof}

Now, given the original configuration of the board, let $A_i$ be the number of chips on square $i$.

\end{proof}
\end{document}
