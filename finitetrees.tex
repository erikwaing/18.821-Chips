\documentclass[11pt]{article}
\usepackage{graphicx}    % needed for including graphics e.g. EPS, PS
\topmargin -1.5cm        % read Lamport p.163
\oddsidemargin -0.04cm   % read Lamport p.163
\evensidemargin -0.04cm  % same as oddsidemargin but for left-hand pages
\textwidth 16.59cm
\textheight 21.94cm 
%\pagestyle{empty}       % Uncomment if don't want page numbers
\parskip 7.2pt           % sets spacing between paragraphs
%\renewcommand{\baselinestretch}{1.5} % Uncomment for 1.5 spacing between lines
\usepackage{amsmath}
\usepackage{amsthm}
\usepackage{amsfonts}
\usepackage{verbatim}
\newtheorem{theorem}{Theorem}
\newtheorem{lemma}{Lemma}
\newtheorem{definition}{Definition}
\parindent 0pt		 % sets leading space for paragraphs
\author{Perry Kleinhenz, Fermi Ma, and Erik Waingarten}
\title{18.821 - Poker Chips Finite board configurations}


\begin{document}

\maketitle

\section{$n$-Trees}
Consider a semi-infinite $n$-tree. That is the graph with a parent node which has $n$ unique children. Those children each have $n$ unique children and so on. 

We can join two semi infinite $n$-trees by an edge connecting their parent nodes. This forms a graph where every vertex has $n+1$ neighbors. We would like to adapt our description of chips and earthquakes to this graph. 

\begin{definition} We define a coordinate system for our graph, with the following rules:
\begin{enumerate}
	\item $(0,0)$ and $(1,0)$ are valid coordinates and refer to the two parent nodes.
	\item $(a,b)$ is a valid coordinate for $a<0$ if $0<b \leq n^{|a|}$
	\item $(a,b)$ is a valid coordinate for $a>1$ if $0<b \leq n^{|a-1|}$
\end{enumerate}
We define the set of all valid coordinates $G_n$ as 
\begin{equation}
G_n := \{ (a,b) \in \mathbb{Z} \times \mathbb{Z} ; (a,b) \text{ obey the rules above}\}
\end{equation}
\end{definition}

\begin{definition}
We define a tree to be a map from the set of valid coordinates to the nonnegative integers, that is 
\begin{equation}
f:  G\rightarrow \mathbb{Z}_{\geq 0}
\end{equation}
We write $\mathcal{T}_n$ to refer to the set of all joined $n$-tree's 
\end{definition}

\begin{definition}
\label{bigearthquakedeftree}
We define the big earthquake operator as  as a map from the set of all boards to the set of all boards and write $BE: \mathcal{T}_n \rightarrow \mathcal{T}_n$. 
It transforms all of the nodes on the tree with the following rule.
\begin{equation}
BE(f(x,y)) = f(x,y) - (n+1)[[f(x,y) \geq n+1]] + [[f(x-k,y_0) \geq n+1]] +  \sum_{i=1}^{n} [[f(x-k,y_i) \geq n+1]] 
\end{equation} 
where $k$ is defined as 
\begin{equation}
k = \begin{cases}
1 \text{ if } x\leq 0 \\
-1 \text{ if } x \geq 1
\end{cases}
\end{equation}
and $(x-k, y_j)$ refers to the $j$th child of $(x,y)$ and $[[A]]$ is defined as 
\begin{equation}
[[A]] = 
\begin{cases} 
1 \text{ if } A \text{ is true} \\ 
0 \text{ if } A  \text{ is false}
 \end{cases}.
\end{equation}
\end{definition}

\begin{definition}
 We define the small earthquake operator as a map from the set of all $n$-trees to the set of all $n$-trees and write $SE: \mathcal{T}_n \rightarrow \mathcal{T}_n$.
It selects one valid node $(x,y)$ with $f(x,y) \geq n+1$ and transforms it and all of its neighbors with the following rule: 
\begin{align*}
SE( f( x, y ) ) = f( x , y )-n-1 \\
SE( f( x + k, y)) = f( x + k, y )+1 \\
SE( f(x-k, y_i) ) = f(x-k, y_j )+1 \\
\end{align*}
where $k$ is as defined above and $(x-k, y_j)$ refers to the $j$th child of $(x,y)$
and leaves all other squares unchanged.
\end{definition}

\begin{definition}
We say that a $n$-Tree is stable if its values are strictly less than $n+1$. This is equivalent to both of the earthquake maps having no effect on the board. 
\end{definition}

\begin{definition} 
We say that a board is finite if the total number of chips on it is finite. That is a board $f$ is finite if the sum
\begin{equation}
S= \sum_{(x,y) \in G_n} f(x,y) 
\end{equation}
is finite. 
We denote the set of finite boards as $\mathcal{T}_n^f$.
\end{definition}

As it turns out we can use a  method of proof nearly identical to the standard board case to show that the state of a tree under this process can never cycle.

We once again define a potential function 
\begin{definition} Let $\Phi: \mathcal{T}_n^f \rightarrow \mathbb{Z}_{\geq 0}$ be the  function
\begin{equation}
\Phi(f) = \sum_{(x,y) \in G_n} f(x,y)(|x|)^2
\end{equation}
Note that this is well defined for trees which contain a finite number of chips. 
\end{definition} 

\begin{lemma}
If $SE(f) \neq f$, then $SE^n(f) \neq f$ for all $n > 0$.
\end{lemma}

\begin{proof}
We will show that  if $SE$ acts on a board nontrivially then $\Phi(SE(f))$ will be strictly larger. That is  
\begin{equation}
SE(f) \neq f \Rightarrow \Phi(SE(f)) > \Phi(f)
\end{equation}


We can evaluate $\Phi(SE(f)) - \Phi(f)$.
\begin{align}
\Phi(SE(f))-\Phi(f) = \sum_{(x,y) \in G_n} SE(f)(x,y)(x)^2 - \sum_{(x,y) \in G_n} f(x,y)(x)^2 
\end{align}
Only one node call it $(x_0,y_0)$ and its neighbors are affected. Therefore the above expression simplifies to
\begin{align*}
\Phi(SE(f)-\Phi(f) = SE( f(x_0,y_0) )* (x_0)^2 + SE( (f(x_0+k, y_0) )* (|x_0+k|)^2 \\
+ \sum_{i} SE( f(x_0-k, y_i) ) * (|x_0-k|)^2 \\
= -(n+1)*(|x_0|)^2 + (|x_0+k)^2 + n (|x_0-k|)^2 \\
\geq-(n+1)*(|x_0|)^2 + (|x_0|-1)^2 + n (|x_0|+1)^2
= -nx^2-x^2+x^2-2|x|+1+n(x^2+2|x|+1) \\
= 2x(n-1)+n+1 > 0
\end{align*}
Where $(x_0-k,y_i)$ refers to the $j$th child of $(x_0,y_0)$ and $k$ is as defined in Definition \ref{bigearthquakedeftree}.
\end{proof}

\begin{lemma}
If $BE(f) \neq f$, then $BE^n(f) \neq f$ for all $n > 0$.
\end{lemma}

\begin{proof}
We will show that 
\begin{equation}
BE(f) \neq f \Rightarrow \Phi(BE(f)) > \Phi(f)
\end{equation}
This means that if there is an earthquake that changes things, then $\Phi(f)$ increases, which means that if $E(f) \neq f$, then for all $N > 0$, $E^N(f) \neq f$, or that you never return to your original state.

We can evaluate $\Phi(BE(f)) - \Phi(f)$.
\begin{align}
\Phi(BE(f))-\Phi(f) = \sum_{(x,y) \in G_n} BE(f)(x,y)(|x|)^2 - \sum_{(x,y) \in G_n} f(x,y)(|x|)^2 
\end{align}
Let $A$ be the set of active locations, or locations with $f(x,y) \geq (n+1)$, which is non-empty by assumption. 
\begin{align*}
\Phi(BE(f)) - \Phi(f) =  \sum_{(x,y) \in A}  BE(f(x,y))x^2 + BE(f(x+k,y))(|x+k|)^2 + \\
\sum_{i} \left( BE(f(x-k,y_i))(|x-k|)^2 \right) -f(x,y)x^2 +f(x+k,y) (|x+k|)^2 +  \sum_{i} \left ( f(x-k,y_i) (|x-k|)^2 \right)  \\
=\sum_{(x,y) \in A} -(n+1)x^2 + (|x+k|)^2 + n(|x-k|)^2 
\end{align*}
Where $(x-k,y_i)$ refers to the $j$th child of $(x,y)$ and $k$ is as defined in Definition \ref{bigearthquakedeftree}. We note that each term in this sum is positive by our proof of the lemma for small earthquakes and so the sum itself must be positive. That is 
\begin{equation}
\Phi(BE(f)) - \Phi(f) > 0
\end{equation}

\end{proof}

\begin{lemma}
\label{finiteextensiontree}
Suppose $\sum_{x,y} f(x,y) = S$, and all chips are between the levels $[x_l, x_h]$. Then $BE^m(f)$ and $SE^m(f)$ only contain chips within the range $[x_l - 2S, x_h + 2S]$ for all $m$.
\end{lemma}

\begin{proof}
It is obvious that a board with $S$ chips on level $l<0$ will have a lower $X_{min}$ than a board with $S$ chips on level 0, and the analogous statement for $X_{max}$ holds as well. 

Now as a very crude lower bound we know that $S$ chips on a single node at level $l<0$ will have an $X_{min}$ no lower than $l-2S$ as there must be at least one chip on every other level, as chips only move one level at a time. 

If we take any other arrangement of $S$ chips, with $X_{min}=l$ but not all $S$ chips are on a single node. We know that $X_{min}$ for this board is bounded below by $l-2S$. In order to have an $X_{min}$ below this at some point in time it would be possible to have $S$ chips on the same node on level $l$. But this cannot occur because in order to move a chip down a level, one chip must move up a level. Therefore if $X_{min}=x_l$, both $BE^m(f)$ and $SE^m(f)$ do not contain chips below level $x_l -2S$.

An analogous argument shows that if $X_{max}=x_h$, both $BE^m(f)$ and $SE^m(f)$ do not contain chips above level $x_h+2S$.
\end{proof}

So now we are ready to prove the theorem that all finite boards will reach a stable state. 

\begin{proof}
(of Theorem~\ref{finitestability}) Suppose $f \in \mathcal{B}$ is a finite board, so $\sum_{x,y} f(x,y) = S$, and suppose all chips lie between levels  $[x_l, x_h]$. Then by Lemma ~\ref{finiteextensiontree}, we have that for all $m$ all chips in $BE^m(f)$ and $SE^m(f)$ will lie in $R = [x_l - 2S, x_h + 2S]$

We have restricted the size of the board to be finite, and the number of chips that can be placed on the board to be finite. 
Thus the total number of ways that the chips can be arranged on the board is finite. 
In particular there are fewer than $(x_h - x_l+1 + 4 S)$ levels and there are at most 
\begin{equation}
n^d,
\end{equation}
nodes at each level, where $d= \max(x_h, |x_l|)$. So there are at most 
\begin{equation}
N:=(x_h - x_l+1 + 4 S)n^d
\end{equation}
nodes that are potentially occupied. Therefore there are at most 
\begin{equation}
\binom{N+S-1}{N-1}
\end{equation}
possible boards $b \in \mathcal{T}_n^f$ that satisfy this. We know that there are no cycles in the states, so there must be an $0 \leq N < \binom{N+S-1}{N-1}$ such that $E^N(f) = E^{N+m}(f)$ for all $m \geq 0$.
\end{proof}


\end{document}
