\documentclass[11pt]{article}
\usepackage{graphicx}    % needed for including graphics e.g. EPS, PS
\topmargin -1.5cm        % read Lamport p.163
\oddsidemargin -0.04cm   % read Lamport p.163
\evensidemargin -0.04cm  % same as oddsidemargin but for left-hand pages
\textwidth 16.59cm
\textheight 21.94cm 
%\pagestyle{empty}       % Uncomment if don't want page numbers
\parskip 7.2pt           % sets spacing between paragraphs
%\renewcommand{\baselinestretch}{1.5} % Uncomment for 1.5 spacing between lines
\usepackage{amsmath}
\usepackage{amsthm}
\usepackage{amsfonts}
\usepackage{verbatim}
\usepackage{graphicx}
\newtheorem{theorem}{Theorem}
\newtheorem{lemma}{Lemma}
\newtheorem{definition}{Definition}
\newtheorem{observation}{Observation}

\parindent 0pt		 % sets leading space for paragraphs
\author{Perry Kleinhenz, Fermi Ma, and Erik Waingarten}
\title{18.821 - Poker Chips infinite board configurations}


\begin{document}         
\maketitle

The proof will be as follows: first, we will show that finite boards cannot return to a previous state. Let $B$ be the set of all possible board configurations, that is, $B = \{ f: \mathbb{Z} \times \mathbb{Z} \rightarrow \mathbb{Z}^{\geq 0}\}$. Each board configuration is given by a function of the integer lattice to the nonnegative integers. So $f(x,y)$ is the number of chips in the grid location $x,y$. 

The earthquake is defined as $E: B \rightarrow B$ acts in the following manner:
\[ E(f(x,y)) = f(x,y) + \sum_{l \in n(x,y)} [[ f(l) \geq 4 ]] - 4[[f(x,y) \geq 4]] \]
Where $n(x,y)$ is the set of neighbors of $x,y$, and $[[ X ]] = 1$ if $X$ happened and $0$ otherwise.

\begin{observation}
There exists boards $f \in B$ with $E(f) \neq f$, but there exists some $n > 0$, $E^n(f) = f$.
\end{observation}
Take
\[ f = \begin{array}{cccccc} \ddots & \vdots & \vdots & \vdots & \vdots & \reflectbox{$\ddots$} \\
				      \dots & 0 & 4 & 0 & 4 & \dots \\
				      \dots & 4 & 0 & 4 & 0 & \dots \\
				      \dots & 0 & 4 & 0 & 4 & \dots \\
				      \reflectbox{$\ddots$} & \vdots & \vdots & \vdots & \vdots & \ddots \end{array} \]
Note that $E(f) \neq f$ but $E^2(f) = f$.
					

\end{document}
